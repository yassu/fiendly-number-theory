\documentclass{jarticle}
\newcommand{\cnt}[1]{\underline{\textbf{#1}}}
\usepackage{enumitem}
\usepackage{listing, jlisting}
\usepackage{yassu}

\title {「はじめての数論」の回答例}
\author{yassu}
\date{\today}
\begin{document}
\maketitle

\section{第1章の回答例} %{{{

\cnt{1.1} %{{{
以下のプログラムによって, 3, 4番目の三角数は $1225 = 35^2$, $416 = 204^2$.

% \lstinputlisting{./../rust/src/ex/ex1/ex1_1.rs}
% TODO: 上のファイルを 挿入する
有効な方法は分からない.

こんな数はたぶん無限にある.
%}}}

\cnt{1.2} %{{{

\begin{flalign*}
1            & = 1^2, \\
1 + 3   = 4  & = 2^2, \\
4 + 5   = 9  & = 3^2, \\
9 + 7   = 16 & = 4^2, \\
16 + 9  = 25 & = 5^2
\end{flalign*}
などとなるから
\[
  \sum_{j=1}^n (2j-1) = n^2
\]
が予想される.
% TODO: 奇数の和の図を書く.

%}}}

\cnt{1.3} %{{{

三つ子素数は $(3, 5, 7)$に限ることを示す.

任意の自然数は $3l, 3l+1, 3l+2 (l \in \NN)$と表すことができる.

$(p, p+2, p+4)$を三つ子素数とする.
このとき, ある自然数$l$があって, $p=3l, 3l+1, 3l+2$のいずれかで表される.

$l=1$のとき, $(p, p+2, p+4) = (3, 5, 7)$ である.

$l \neq 1$のとき, $3l$は素数ではないから, $p=3l+1$ もしくは $3l+2$と表される.

$p = 3l+1$とすると, $p + 2 = 3l+3 = 3(l + 1)$となって, これは素数ではないから不敵.

$p = 3l+2$とすると, $p + 2 = 3l+4, p + 4 = 3l+6 = 3(l + 2)$となるから, $p+4$は素数ではない.

以上によって, 三つ子素数は$(3, 5, 7)$に限る.

%}}}

%}}}




\end{document}
