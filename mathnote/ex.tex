\documentclass{jarticle}
\newcommand{\ccheckmark}{\checkmark \checkmark}
\newcommand{\cnt}[1]{\noindent \underline{\textbf{#1}}}
% LCM
\makeatletter
\def\LCM{\mathop{\operator@font LCM}}
\makeatother
\usepackage{longtable}
\usepackage{enumerate}
\usepackage{listing, jlisting}
\usepackage{yassu}

\title {「はじめての数論」の回答例}
\author{yassu}
\date{\today}
\begin{document}
\maketitle

\section{第1章の回答例} %{{{

\cnt{1.1} %{{{
以下のプログラムによって, 3, 4番目の三角数は $1225 = 35^2$, $416 = 204^2$.

% \lstinputlisting{./../rust/src/ex/ex1/ex1_1.rs}
% TODO: 上のファイルを 挿入する
有効な方法は分からない.

こんな数はたぶん無限にある.
%}}}

\cnt{1.2} %{{{

\begin{flalign*}
1            & = 1^2, \\
1 + 3   = 4  & = 2^2, \\
4 + 5   = 9  & = 3^2, \\
9 + 7   = 16 & = 4^2, \\
16 + 9  = 25 & = 5^2
\end{flalign*}
などとなるから
\[
  \sum_{j=1}^n (2j-1) = n^2
\]
が予想される.
% TODO: 奇数の和の図を書く.

%}}}

\cnt{1.3} %{{{

三つ子素数は $(3, 5, 7)$に限ることを示す.

任意の自然数は $3l, 3l+1, 3l+2 (l \in \NN)$と表すことができる.

$(p, p+2, p+4)$を三つ子素数とする.
このとき, ある自然数$l$があって, $p=3l, 3l+1, 3l+2$のいずれかで表される.

$l=1$のとき, $(p, p+2, p+4) = (3, 5, 7)$ である.

$l \neq 1$のとき, $3l$は素数ではないから, $p=3l+1$ もしくは $3l+2$と表される.

$p = 3l+1$とすると, $p + 2 = 3l+3 = 3(l + 1)$となって, これは素数ではないから不敵.

$p = 3l+2$とすると, $p + 2 = 3l+4, p + 4 = 3l+6 = 3(l + 2)$となるから, $p+4$は素数ではない.

以上によって, 三つ子素数は$(3, 5, 7)$に限る.

%}}}

\cnt{1.4} % {{{
% TODO: (a), (b)をenumerateに変更する
(a) $N^2 - 1$が十数であるのは $N=2$のときに限る. % {{{
なぜなら,
\[
  N^2 - 1 = (N - 1) (N + 1)
\]
であり, $N^2 - 1$ が十数であるためには $N-1 = 1$ となる必要があるからである.
% }}}

(b) おそらく無数に存在する. % {{{
$N$によってそれぞれ調べてみると
% TODO: 以下をtabularを使ってきれいに書く

\noindent $N = 2 \then N^2 - 2 = 2$; 素数, \\
\noindent $N = 3 \then N^2 - 2 = 7$; 素数, \\
\noindent $N = 4 \then N^2 - 2 = 14 = 2 \times 7$, \\
\noindent $N = 5 \then N^2 - 2 = 23$; 素数, \\
\noindent $N = 6 \then N^2 - 2 = 34 = 2 \times 17$, \\
\noindent $N = 7 \then N^2 - 2 = 47$; 素数,\\
\noindent $N = 8 \then N^2 - 2 = 62 = 2 \times 31$, \\
\noindent $N = 9 \then N^2 - 2 = 79$; 素数, \\
\noindent $N = 10 \then N^2 - 2 = 98 = 2 \times 7^2$, \\
\noindent $N = 11 \then N^2 - 2 = 119$, \\
\noindent $N = 12 \then N^2 - 2 = 142 = 2 \times 71$, \\
\noindent $N = 13 \then N^2 - 2 = 167$, \\
\noindent $N = 14 \then N^2 - 2 = 194 = 2 \times 97$, \\
\noindent $N = 15 \then N^2 - 2 = 253$. % }}}

(c) $N^2 - 3$の形の素数は多分無数に存在する. \\ % {{{
最初の方から正の数を列挙してみると

% TODO: 以下の表をtabularを使ってきれいに書く(noindentにすること)
$N = 2 \then 2^2 - 3 = 1$, \\
$N = 3 \then 3^2 - 3 = 6 = 2 \times 3$, \\
$N = 4 \then 4^2 - 3 = 13$; 素数, \\
$N = 5 \then 5^2 - 3 = 22 = 2 \times 11$, \\
$N = 6 \then 6^2 - 3 = 33 = 3 \times 11$, \\
$N = 7 \then 7^2 - 3 = 46 = 2 \times 23$, \\
$N = 8 \then 8^2 - 3 = 61$; prime, \\
$N = 9 \then 9^2 - 3 = 78 = 2 \times 39$, \\
$N = 10 \then 10^2 - 3 = 97$; prime

$N^2 - 4$の形の素数は5に限る.
なぜなら
\[
  N^2 - 4 = (N-2)(N+2)
\]
となるからである. %}}}

(d) 少なくとも平方数ではない.

% }}}

\cnt{1.5} % {{{

\noindent $n$が偶数のとき: \\
$n = 2m$ とおくと
\begin{align*}
1 + 2 + \cdots + 2m
  &= (2m + 1) + ((2m - 1) + 2) + ((2m - 2) + 3) + \cdots + ((m + 1 + m)) \\
  &= m \cdot (2m+1) \\
  &= \frac{n}{2}(n+1).
\end{align*}

\noindent $n$が奇数のとき: \\
$n = 2m - 1$とおくと
\begin{align*}
1 + 2 + \cdots + (2m - 1)
  &= (1 + (2m - 1)) + (2 + (2m - 2)) + \cdots + ((m - 1) + (m + 1)) + m \\
  &= \underbrace{2m + 2m + \cdots + 2m}_{m-1} + m \\
  &= 2m(m-1) + m \\
  &= (n+1)(\frac{n+1}{2} - 1) + \frac{n+1}{2} \\
  &= (n+1)(\frac{n+1}{2} - 1 + \frac{1}{2}) \\
  &= (n+1) \cdot \frac{n+1 - 2 + 1}{2} = \frac{n}{2}(n+1).
\end{align*}

% }}}

%}}}

\section{第2章の回答例} % {{{

\noindent \cnt{2.1} \indent % {{{
% TODO: 下の(a), (b)でenumerateするようにする
\noindent (a) 組み合わせとして考えるのは
\begin{align*}
(a, b) =& (3m, 3n), (3m+1, 3n), (3m+2, 3n), \\
        & (3m, 3n+1), (3m+1, 3n+1), (3m+2, 3n+1), \\
        & (3m, 3n+2), (3m+1, 3n+2), (3m+2, 3n+2)
\end{align*}
のように書かれる場合である.
必要なら$a$と$b$を入れ替えることによって, この表の右上半分だけを考える.
すなわち,
\begin{align*}
(a, b) = (3m+1, 3n+1), (3m+2, 3n+1), (3m+2, 3n+2)
\end{align*}
の場合 対応する既約ピタゴラス数$(a, b, c)$が存在しないことを示したい.

まず, $c = 3l$とかけているとき, $c^2$は3の倍数であり,
$c = 3l+1$とかけているとき, $c^2$は3で割ると1余り,
$c = 3l+2$とかけているとき, $c^2$は3で割ると1余る.

\noindent $(a, b) = (3m+1, 3n+1)$ のとき,
\begin{align*}
a^2 + b^2
  &= (3m+1)^2 + (3n+1)^2 \\
  &= 9(m^2 + n^2) + 6(m+n) + 2
\end{align*}
となるが, $c^2$は3で割ると2余る組がないから不適.

\noindent $(a, b) = (3m+2, 3n+1)$のとき,
\[
  a^2 + b^2 = 9(m^2 + n^2) + 6(2m + n) + 5.
\]
よって, この場合も$a^2 + b^2$を3で割ると2余るので不適.

\noindent $(a, b) = (3m+2, 3n+2)$のとき
\[
  (3m+2)^2 + (3n+2)^2 = 9(m^2 + n^2) + 12(m+n) + 8.
\]
よってこの場合も$a^2 + b^2$を3で割ると2余るので不適.

(b) 分からない.
% }}}

\cnt{2.2} % {{{
仮定より, ある整数$k_1, k_2$があって
\begin{align*}
  m &= dk_1, \\
  n &= dk_2
\end{align*}
が成り立つ.
このとき,
\begin{align*}
  m + n &= d(k_1 + k_2), \\
  m - n &= d(k_1 - k_2)
\end{align*}
となるから, 主張を得る.
% }}}

\cnt{2.3} \indent % {{{
% 以下の(a), (b)などはenumerateで書き換えること
(a) 任意の1より大きな奇数が現れる.
実際, 定理 2.1で$t=1$とおけば $a = s$.

(b) 少なくとも 4の倍数が現れる. 8で割って4余る数はかならず現れる.

$b$は4の倍数であることを示す.

$s, t$はともに奇数であるから, ある自然数$s_1, t_1$ があって, $s = 2s_1 - 1$, $t = 2t_1 - 1$
が成り立つ.
このとき
\begin{align*}
b
  &= \frac{1}{2}(s - t)(s + t) \\
  &= \frac{1}{2}(2s_1 - t_1)(2s_1 + 2t_1 - 2) \\
  &= 2(s_1 - t_1)(s_1 + t_1 - 1)
\end{align*}
となる.
ここで, $s_1 - t_1$ と $s_1 + t_1 - 1$は2を法として異なるので, $b$ は4の倍数である.

次に, 8で割って4余る数は必ず現れることを示す.

$b$を4で割って8余る自然数とする. ある互いに素な奇数$s, t (s > t)$ があって
\[
  b = \frac{s^2 - t^2}{2}
\]
を満たすことを示したい.
$b = 4k$($k$は奇数) と置くと
\[
  4k = \frac{s^2 - t^2}{2}.
\]
分母を払って
\[
  8k = s^2 - t^2.
\]
ここで
\begin{align*}
s^2
  &= t^2 + 8k \\
  &= t^2 + 8(k-2) + 16
\end{align*}
であるから, $t = (k-2)^2$と置くと
\begin{align*}
  s^2
    &= (k-2)^2 + 8(k-2) + 16 \\
    &= (k + 2)^2.
\end{align*}
よって, $s = k + 2$と置けばよい.
最後に, $k$は奇数であったから, $k-2$ と $k+2$ は互いに素である.

(c) 4で割ると1余る数.

$s = 2s_1 - 1$, $t = 2s_2 - 1$ と置くと
\begin{align*}
c
  &= \frac{s^2 + t^2}{2} \\
  &= \frac{1}{2} ((2s_1 - 1)^2 + (2s_2 - 1)^2) \\
  &= \frac{1}{2} (8(s_1^2 + s_2^2) - 8s_1s_2 + 2) \\
  &= 4(s_1^2 + s_2^2) - 4s_1s_2 + 1 \\
  &= 4(s_1^2 - s_1s_2 + s_2^2) + 1
\end{align*}
となるから, $c$は4で割ると1余る.
% }}}

\cnt{2.4} % {{{
見つけられない. % }}}

\cnt{2.5} % {{{
% 以下の a), b) をenumerate にすること.
\noindent a) % {{{
$n \le 4$における$a, b, c$の値を表にすると
% TODO:  以下の表をtabularを使うこと.
\begin{align*}
T_1 & = 1         & b_1 &= 4 \cdot 1  = 4 & 3^2 + 4^2  &= 5^2, \\
T_2 & = 1 + 2 = 3 & b_2 &= 4 \cdot 3 = 12 & 5^2 + 12^2 &=13^2, \\
T_3 & = 3 + 3 = 6 & b_3 &= 4 \cdot 6 = 24 & 7^2 + 24^2 &= 25^2, \\
T_4 & = 6 + 4 = 10 & b_4 &= 4 \cdot 10 = 40 & 9^2 + 40^2 &= 41^2.
\end{align*}
ここで, $b_n$を$T_n$に対応する既約ピタゴラス数の$b$とおいた.

\noindent $T_5)$ $T_5 = 10 + 5 = 15$. また, $b_5 = 4 \cdot 15 = 60$. % {{{

これより, $11^2 + 60^2 = 61^2$が予想できる.
実際計算してみると
\[
  11^2 + 60^2 = 121 + 3600 = 3721 = 61^2.
\]
% }}}

\noindent $T_6)$ % {{{
$T_6 = 15 + 6 = 21$, $b_6 = 4 \cdot 21 = 84$ より
\[
  13^2 + 84^2 = 85^2
\]
が予想できる.
計算してみると
\begin{align*}
13^2 + 84^2
  &= 169 + 7059 \\
  &= 7225, \\
85^2 &= 7225
\end{align*}
であるから, 確かに成り立っている.
% }}}

\noindent $T_7)$ % {{{
$T_7 = 21 + 7 = 28$, $b_7 = 4 \cdot 28 = 112$ であるから
\[
  15^2 + 112^2 = 113^2
\]
が予想される.
実際
\begin{align*}
15^2 + 112^2
  &= 225 + 12544 \\
  &= 12769, \\
113^2 &= 12769
\end{align*}
なので, 確かに成り立っている.
% }}}
% }}}

\noindent b) % {{{
存在する.
\[
  (a, b, c) = (2n+1, 4T_n, 4T_n + 1)
\]
が既約ピタゴラス数の組であることを示す.
実際, 定理 2.1において $t=1, s=2n+1$ とおけば $t$ と $s$ は互いに素で
\begin{align*}
a &= 2n + 1, \\
b
  &= \frac{(2n+1)^2 - 1}{2} \\
  &= \frac{4n^2 + 4n}{2} \\
  &= 2n(n+1) \\
  &= 4 \cdot \frac{n}{2} (n+1) = 4T_n, \\
c
  &= \frac{1^2 + (2n+1)^2}{2} \\
  &= \frac{4n^2 + 4n + 2}{2} \\
  &= 2n^2 + 2n + 1 \\
  &= 4 \cdot \frac{n}{2} (n+1) + 1 = 4T_n + 1.
\end{align*}
% }}}

% }}}

\cnt{2.6} % {{{

定理 2.1 において $c - a = 2$ とおけば
\[
  c - a = \frac{s^2 + t^2}{2} - st = 2
\]
より
\[
  s = t + 2
\]
を得る.

% TODO: 以下の (a), (b) をenumerateに

\noindent (a) % {{{

$t = 1, s = 3$ とおけば
\[
  a = 3, \quad
  b = \frac{9 - 1}{2} = 4, \quad
  c = \frac{3^2 + 1^2}{2} = 5.
\]
また, $t=3, s=5$ とおけば
\[
  a = 15,
  b = \frac{5^2 - 3^2}{2} = \frac{16}{2} = 8,
  c = \frac{3^2 + 5^2}{2} = 17.
\]
% }}}

\noindent (b) % {{{

差が2である自然数の組が共通因数を持つには, その2つの自然数が偶数であり,
共通因数は2である場合にかぎる.

例えば, $s = 999, t = 997$ とおくと
\[
  a = 999 \cdot 997,
  b = \frac{999^2 - 997^2}{2},
  c = \frac{999^2 * 997^2}{2}
\]
を得て, $c - a = 2$ である.
% }}}

\noindent (c) % {{{
$s = t + 2$ を定理 2.1に代入すればよい.
すなわち
\begin{align*}
  a &= t \cdot (t+2), \\
  b
    &= \frac{(t+2)^2 - t^2}{2} \\
    &= \frac{4t+4}{2} \\
    &= 2(t+1), \\
c
  &= \frac{t^2 + (t + 2)^2}{2} \\
  &= \frac{2t^2 + 4t + 4}{2} \\
  &= t^2 + 2t + 2.
\end{align*}
ここで, $t$は奇数.
% }}}

% }}}

\cnt{2.7} % {{{
定理 2.1の下の表に $2c - 2a = 2(c-a)$ の値を追記すると以下のようになる.

\begin{tabular}{llllll}
s & t & a & b & c & 2(c-a) \\
3 & 1 & 3 & 4 & 5 & 4 \\
5 & 1 & 5 & 12 & 13 & 8 \\
7 & 1 & 7 & 24 & 25 & 36 \\
9 & 1 & 9 & 40 & 41 & 64 \\
5 & 3 & 15 & 8 & 17 & 4 \\
7 & 3 & 21 & 20 & 29 & 16 \\
7 & 5 & 35 & 12 & 37 & 4 \\
9 & 5 & 45 & 28 & 53 & 16 \\
9 & 7 & 63 & 15 & 65 & 4
\end{tabular}

この表によって $2(c-a) = (s-t)^2$ が予想される.
実際
\begin{align*}
2 \cdot (c-a)
  &= 2\cdot (\frac{s^2 + t^2}{2} - st) \\
  &= s^2 + t^2 - 2st \\
  &= s^2 - 2st + t^2 \\
  &= (s-t)^2
\end{align*}
であるから, 主張は正しい.
% }}}
% }}}

\section{第3章の回答例} % {{{

\cnt{3.1} % {{{

\noindent (a) % {{{
$u$と$v$が共通因数$k > 1$をもつとする.
このとき, $u = ku^\prime, v = kv^\prime$とおくと
\[
  a = k^2(u^{\prime 2} - v^{\prime 2}), \quad
  b = 2k^2 u^{\prime 2} v^{\prime 2}, \quad
  c = k^2 (u^{\prime 2} - v^{\prime 2})
\]
となるから, $a, b, c$は共に$k^2$で割り切れる.
% }}}

\noindent (b) % {{{
$u$, $v$が共に奇数であれば, $u^2 - v^2$, $u^2 + v^2$は偶数であるから, $(a, b, c)$は
  既約ピタゴラス数の組でない.

$u=3, v=1$とおくと
\[
  (a, b, c) = (8, 6, 10)
\]
となるが, これは既約ピタゴラス数の組ではない.
% }}}

\noindent (c) % {{{
(d)の問題のために, $s$と$t$が共通因数を持つ場合は右に$\checkmark$を,
$s$と$t$が共通因数を持たないが$(a, b, c)$が既約ピタゴラス数の組ではない場合は$\ccheckmark$をつける.

\begin{longtable}{llllll}
u & v & a & b & c & \\
2 & 1 & 3 & 4 & 5 & \\
3 & 1 & 8 & 6 & 10 & \ccheckmark \\
3 & 2 & 5 & 12 & 13 & \\
4 & 1 & 15 & 8 & 17 & \\
4 & 2 & 12 & 16 & 20 & \checkmark \\
4 & 3 & 7 & 24 & 25 & \\
5 & 1 & 24 & 10 & 26 & \ccheckmark \\
5 & 2 & 21 & 20 & 29 & \\
5 & 3 & 16 & 30 & 34 & \ccheckmark \\
5 & 4 & 9 & 40 & 41 & \\
6 & 1 & 35 & 12 & 37 & \\
6 & 2 & 32 & 24 & 40 & \checkmark \\
6 & 3 & 27 & 36 & 45 & \checkmark \\
6 & 4 & 20 & 48 & 50 & \checkmark \\
6 & 5 & 11 & 60 & 61 & \\
7 & 1 & 48 & 14 & 50 & \ccheckmark \\
7 & 2 & 45 & 28 & 53 & \\
7 & 3 & 40 & 42 & 58 & \ccheckmark \\
7 & 4 & 33 & 56 & 65 & \\
7 & 5 & 24 & 70 & 74 & \ccheckmark \\
7 & 6 & 13 & 84 & 85 & \\
8 & 1 & 63 & 16 & 65 & \\
8 & 2 & 60 & 32 & 68 & \checkmark \\
8 & 3 & 55 & 48 & 73 & \\
8 & 4 & 48 & 64 & 80 & \checkmark \\
8 & 5 & 39 & 80 & 89 & \\
8 & 6 & 28 & 96 & 90 & \checkmark \\
8 & 7 & 15 & 112 & 113 & \\
9 & 1 & 80 & 18 & 82 & \ccheckmark \\
9 & 2 & 77 & 36 & 85 & \\
9 & 3 & 72 & 54 & 90 & \checkmark \\
9 & 4 & 65 & 72 & 97 & \\
9 & 5 & 56 & 90 & 106 & \ccheckmark \\
9 & 6 & 45 & 108 & 117 & \checkmark \\
9 & 7 & 32 & 126 & 130 & \ccheckmark \\
9 & 8 & 17 & 144 & 145 & \\
10 & 1 & 99 & 20 & 101 & \\
10 & 2 & 96 & 40 & 104 & \checkmark \\
10 & 3 & 91 & 60 & 109 & \\
10 & 4 & 84 & 80 & 116 & \checkmark \\
10 & 5 & 75 & 100 & 125 & \checkmark \\
10 & 6 & 64 & 120 & 136 & \checkmark \\
10 & 7 & 51 & 140 & 149 & \\
10 & 8 & 36 & 160 & 164 & \checkmark \\
10 & 9 & 19 & 180 & 181 &
\end{longtable}
% }}}

\noindent (d) % {{{
「$u$と$v$が共通因数を持たない」かつ「$u$と$v$の偶奇が異なる」.
% }}}

\noindent (e) % {{{
$u$と$v$が共通因数を持つ場合, $(a, b, c)$は既約ではないことは 3.1で示した.
$u$と$v$が奇数であるとき, $u^2 - v^2$は偶数であり, $2uv$も偶数であるから, $(a, b, c)$
  は既約でない.

$u$と$v$が共通因数を持たず, $u$と$v$の偶奇が異なるとき $u^2 -
v^2$と$2uv$が互いに素であることを示そう.

$u^2 - v^2$と$2uv$が十数$p$を共通因数として持ったとして, $2uv = pu^\prime, u^2-v^2 =
pv^\prime$とおく.

$u^2 - v^2$は奇数であるから, $p$は奇素数である.
また, $2uv=pu^\prime$より, $u$か$v$は$p$の倍数である.

よって, $u$も$v$も$p$の倍数であり, これは$u$と$v$が共通因数を持たないことに反する.
% }}}

% }}}

\cnt{3.2} % {{{

\noindent (a) % {{{
円の方程式$x^2 + y^2 = 2$に傾き$m \in \QQ$で$(1, 1)$を通る直線の方程式
$y = m(x-1) + 1 = mx - m + 1$を代入すると
\begin{align*}
x^2 + (m(x-1) + 1)^2 = 2 \\
x^2 + (mx - m + 1)^2 = 2 \\
(1 + m^2)x^2 + 2mx(1 - m) + m^2 - 2m - 1 = 0 \\
(x - 1)((1 + m^2)x - (m^2 - 2m - 1)) = 0.
\end{align*}
よって, $x \neq 1$とすると
\[
  x = \frac{m^2 - 2m - 1}{1 + m^2}.
\]
これを $y = mx - m + 1$に代入して
\begin{align*}
y
  &= m \cdot \frac{m^2 - 2m - 1}{1 + m^2} - m + 1 \\
  &= \frac{m(m^2 - 2m - 1) - m (1 + m^2) + 1 + m^2}{1 + m^2} \\
  &= \frac{-m^2 - 2m + 1}{1 + m^2} \\
  &= - \frac{m^2 + 2m - 1}{1 + m^2}.
\end{align*}

最後に $m = \frac{v}{u}$ とおくと
\begin{align*}
  x
    &= \frac{v^2 - 2uv - u^2}{u^2 + v^2}, \\
  y
    &= - \frac{v^2 + 2uv - u^2}{u^2 + v^2} \\
    &= \frac{u^2 - 2uv - v^2}{u^2 + v^2}.
\end{align*}

% }}}

\noindent (b) % {{{
$x^2 + y^2 = 3$上の点$(x_0, y_0) \in \QQ^2$を見つけられない.
% }}}
% }}}

\cnt{3.3} % {{{

$x^2 + y^2 = 1$に$(-1, 0)$を通り $m \in \QQ$を傾きとする直線の方程式$y = m(x+1)$を連立させると
\begin{align*}
x^2 - m^2(x + 1)^2 = 1. \\
(1 - m^2)x^2 - 2m^2x - 1 - m^2 = 0. \\
(x + 1)((1-m^2)x - (1+m^2)) = 0.
\end{align*}

$x \neq -1$ とすると
\[
  x = \frac{1 + m^2}{1 - m^2}.
\]
これを $y = m(x+1)$に代入すると
\begin{align*}
y
  &= m(x+1) \\
  &= m \cdot (\frac{1+m^2}{1-m^2} + 1) \\
  &= m \cdot \frac{1 + m^2 + 1 - m^2}{1 - m^2} \\
  &= \frac{2m}{1 - m^2}.
\end{align*}

また, $m = \frac{v}{u}$とおくと
\[
  x = \frac{u^2 + v^2}{u^2 - v^2}, \quad
  y = \frac{2uv}{u^2 - v^2}.
\]
% }}}

\cnt{3.4} % {{{
TODO
% }}}

% }}}

\section{第4章の回答例} % {{{

\cnt{4.1} % {{{
略
% }}}

\cnt{4.2} \indent % {{{
\noindent (a) $a = xz, b = yz, c = z^2$ とおくと % {{{
\begin{align*}
  x^3 z^3 + y^3 z^3 = z^4 \\
  x^3 + y^3 = z.
\end{align*}
したがって, 例えば
\begin{itemize}
\item $x = 1, y = 2$ のとき $z = 1^3 + 2^3 = 9$ で
\[
  a = 1 \cdot 9 = 9, \quad
  b = 2 \cdot 9 = 18, \quad
  c = 9^2 = 81.
\]
\item $x = 2, y = 3$ のとき $z = 2^3 + 3^3 = 35$ で
\[
  a = 2 \cdot 35 = 70, \quad
  b = 3 \cdot 35 = 105, \quad
  c = 35^2.
\]
\end{itemize}
% }}}

\noindent (b) % {{{
仮定より
\[
  A^3 + B^3 = C^2.
\]
このとき
\begin{align*}
(n^2A)^3 + (n^2B)^3
  &= n^6(A^3 + B^3) \\
  &= n^6 C^2 \\
  &= (n^3C)^2.
\end{align*}
% }}}

\noindent (c) % {{{

(a)の$x, z$が平方因子を持たず かつ互いに素であれば $a = xz$ に平方因子は現れないから
$(a, b, c)$ は $(b)$の意味の既約である.

以下のステップで解を見つける:

\begin{enumerate}[(i)]
\item $x = 1, y$を適当な平方因子を持たない自然数として $z = x^3 + y^3$ を計算する.
\item $z$ が平方因子を持たなければ $(a, b, c) = (xz, yz, z^2)$ は 既約である
  \footnote{xzに平方因子が現れないため}.
\end{enumerate}

$x=1, y=2$ のとき) $z = 1^3 = 9^3 = 9 = 3^2$. これは平方因子を持つ.

$x=1, y=3$ のとき) $z = 1^3 + 3^3 = 28 = 2^2 \cdot 7$. これは平方因子を持つ.

$x=1, y=4$ のとき) $z = 1^3 + 4^3 = 65 = 5 \times 13$. これは平方因子を持たない.

$x=1, y=5$ のとき) $z = 1^3 + 5^3 = 126 = 2 \times 63 = 2 \times 7 \times 3^2$.
  これは平方因子を持つ.

$x=1, y=6$のとき) $z = 1^3 + 6^3 = 217 = 7 \times 31$. これは平方因子を持たない.

$x=1, y=7$のとき)
\begin{align*}
z
  &= 1^3 + 7^3 \\
  &= 344 \\
  &= 2^2 \times 86.
\end{align*}
これは平方因子を持つ.

$x = 1, y=8$のとき)
\begin{align*}
z
  &= 1^3 + 8^3 \\
  &= 513 \\
  &= 3^2 \times 57.
\end{align*}
これは平方因子を持つ.

$x = 1, y = 9$ のとき $z = 1^3 + 9^3 = 730 = 2 \times 3 \times 73$. これは平方因子を持たない.

以上によって,
\begin{align*}
(a, b, c)
  &= (2, 2, 4), (65, 4 \times 65, 65^2), \\
  & \quad (7 \times 31, 6 \times 7 \times 31, 7^2 \times 31^2),
      (2 \times 3 \times 73, 2 \times 3^3 \times 73, 2^2 \times 3^2 \times 73^2)
\end{align*}
は既約な $(\ast)$ の解である.
% }}}

\noindent (d) % {{{
式 $(\ast)$ で$a = b$とした方程式
\[
  2a^3 = c^2
\]
を考える.
左辺は偶数だから, $c$ も偶数. よって 自然数 $c_1$ があって $c = 2c_1$ と書ける.
このとき
\begin{align*}
2a^3 &= (2c_1)^2 \\
2a^3 &= 4c_1^2 \\
a^3 &= 2c_1^2.
\end{align*}
同様に $a$は偶数なので, 自然数 $a_1$ があって $a = 2a_1$.
このとき
\begin{align*}
  8a_1^3 &= 2c_1^2 \\
  4a_1^3 &= c_1^2.
\end{align*}

同様に$c_1$は偶数なので, 自然数$c_2$があって $c_1 = 2c_2$.
このとき
\begin{align*}
4a_1^3 &= 4c_2^2. \\
a_1^3 &= c_2^2.
\end{align*}
よって ある自然数$n$があって $a_1 = n^2, c_2 = n^3$.
したがって
\begin{align*}
a &= 2a_1 = 2n^2, \\
c &= 2c_1 = 4c_2 = 4n^3.
\end{align*}
以上によって, $(a, b, c) = (2n^2, 2n^2, 4n^3)$($n$は自然数).
% }}}

\noindent (e)
$x = 1$ とし
\[
  1 + y^3 = 1^3 + y^3 > 10^4
\]
とすれば $y^3 > 10^4$ であるが, これをみたす自然数$y$は$y \ge 22$を満たす \footnote{
  $10^{4/3} = 21.544\ldots$のため}

$y=22$ のとき: $z = 1^3 = 22^3 = 10649 = 23 \times 463$ であるから (c)の条件を満たす.
  以上によって
  \[
  (a, b, c) = (23 \times 463, 22 \times 23 \times 463, 23^2 \times 463^2)
  \]
  は$a > 10^4$を満たす$(\ast)$の既約な解である.
% }}}
% }}}

\section{第5章の回答例} % {{{
\begin{enumerate}[5.1] % {{{
\item
\begin{enumerate}[(a)] % {{{
  \item ユークリッドの互除法を用いると
    \begin{align*}
      67890 &= 12345 \times 5 + 6165 \\
      12345 &= 6165 \times 2 + 15 \\
      6165  &= 15 \times 411 + 0
    \end{align*}
    であるから, $\gcd(12345, 67890) = 15$である.
  \item ユークリッドの互除法を用いると
    \begin{align*}
      54321 &= 9876 \times 5   + 4941 \\
      9876  &= 4941 \times 1   + 4935 \\
      4941  &= 4935 \times 1   + 6    \\
      4935  &= 6    \times 822 + 3    \\
    \end{align*}
    であるから, $\gcd(54321, 9876) = 3$.
\end{enumerate} % }}}
\item % {{{
考えているステップは
\begin{align*}
  r_j     &= q_{j+2}r_{j+1} + r_{j+2} \\
  r_{j+1} &= q_{j+3}r_{j+2} + r_{j+3}
\end{align*}
である.
$q_{j+2} \ge 2$ のときは
\begin{align*}
  r_j
    &= q_{j+2}r_{j+1} + r_{j+2} \\
    &\ge 2r_{j+1} + r_{j+2} > 2r_{j+1}
\end{align*}
より $r_{j+1} < \frac{1}{2} r_j$ が成り立つ.
したがって $q_{j+2} = 1$ としてよい.
同様に$q_{j+3} = 1$ としてよい.
このとき, 考えているステップは
\begin{align*}
  r_j &= r_{j+1} + r_{j+2} \\
  r_{j+1} &= r_{j+2} + r_{j+3}
\end{align*}
2式を足すと
\begin{align*}
  r_{j} + r_{j+1} &= r_{j+1} + 2r_{j+2} + r_{j+3} \\
  r_j &= 2r_{j+2} + r_{j+3} > 2r_{j+2}.
\end{align*}
以上によって
\[
  r_{j+2} < \frac{1}{2} r_j.
\]

$2 \log_2(b) \le 7(\log_{10}(b) + 1)$ を示したい.
\begin{align*}
  & 2\log_{2}(b) \le 7(\log_{10}(b) + 1) \\
  \same & 2\log_2(b) \le 7\log_{10}2 \cdot \log_2(b) + 7 \\
  \same & \log_2(b) \cdot (2 - 7\log_{10}(2)) \le 7
\end{align*}
である.
ここで
\[
  10^2 < 2^7
\]
より $2 < \log_{10}2^7 = 7\log_{10} 2$ であるから, $2 - 7\log_{10} 2 < 0$ である.
したがって, 命題が示された.
% }}}
\item TODO
\item
\begin{enumerate}[(a)] % {{{
\item
\begin{enumerate}[(i)] % {{{
\item $8 = 2^3, 12 = 2^2 \cdot 3$ より
  \[
    \LCM(8, 12) = 2^3 \cdot 3 = 24.
  \]
\item $20 = 2^2 \cdot 5, 30 = 2 \cdot 3 \cdot 5$ であるから
  \[
    \LCM(20, 30) = 2^2 \cdot 3 \cdot 5 = 20 \cdot 3 = 60.
  \]
\item $51 = 3 \cdot 17, 68 = 2^2 \cdot 17$ より
  \begin{align*}
    \LCM(51, 68)
      = 2^2 \cdot 3 \cdot 17 &= 4 \cdot 51 \\
                             &= 204
  \end{align*}
\end{enumerate} % }}}
\end{enumerate} % }}}
\end{enumerate} % }}}
% }}}

\end{document}
