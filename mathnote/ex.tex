\documentclass{jarticle}
\newcommand{\cnt}[1]{\noindent \underline{\textbf{#1}}}
\usepackage{enumitem}
\usepackage{enumerate}
\usepackage{listing, jlisting}
\usepackage{yassu}

\title {「はじめての数論」の回答例}
\author{yassu}
\date{\today}
\begin{document}
\maketitle

\section{第1章の回答例} %{{{

\cnt{1.1} %{{{
以下のプログラムによって, 3, 4番目の三角数は $1225 = 35^2$, $416 = 204^2$.

% \lstinputlisting{./../rust/src/ex/ex1/ex1_1.rs}
% TODO: 上のファイルを 挿入する
有効な方法は分からない.

こんな数はたぶん無限にある.
%}}}

\cnt{1.2} %{{{

\begin{flalign*}
1            & = 1^2, \\
1 + 3   = 4  & = 2^2, \\
4 + 5   = 9  & = 3^2, \\
9 + 7   = 16 & = 4^2, \\
16 + 9  = 25 & = 5^2
\end{flalign*}
などとなるから
\[
  \sum_{j=1}^n (2j-1) = n^2
\]
が予想される.
% TODO: 奇数の和の図を書く.

%}}}

\cnt{1.3} %{{{

三つ子素数は $(3, 5, 7)$に限ることを示す.

任意の自然数は $3l, 3l+1, 3l+2 (l \in \NN)$と表すことができる.

$(p, p+2, p+4)$を三つ子素数とする.
このとき, ある自然数$l$があって, $p=3l, 3l+1, 3l+2$のいずれかで表される.

$l=1$のとき, $(p, p+2, p+4) = (3, 5, 7)$ である.

$l \neq 1$のとき, $3l$は素数ではないから, $p=3l+1$ もしくは $3l+2$と表される.

$p = 3l+1$とすると, $p + 2 = 3l+3 = 3(l + 1)$となって, これは素数ではないから不敵.

$p = 3l+2$とすると, $p + 2 = 3l+4, p + 4 = 3l+6 = 3(l + 2)$となるから, $p+4$は素数ではない.

以上によって, 三つ子素数は$(3, 5, 7)$に限る.

%}}}

\cnt{1.4} % {{{
% TODO: (a), (b)をenumerateに変更する
(a) $N^2 - 1$が十数であるのは $N=2$のときに限る. % {{{
なぜなら,
\[
  N^2 - 1 = (N - 1) (N + 1)
\]
であり, $N^2 - 1$ が十数であるためには $N-1 = 1$ となる必要があるからである.
% }}}

(b) おそらく無数に存在する. % {{{
$N$によってそれぞれ調べてみると
% TODO: 以下をtabularを使ってきれいに書く

\noindent $N = 2 \then N^2 - 2 = 2$; 素数, \\
\noindent $N = 3 \then N^2 - 2 = 7$; 素数, \\
\noindent $N = 4 \then N^2 - 2 = 14 = 2 \times 7$, \\
\noindent $N = 5 \then N^2 - 2 = 23$; 素数, \\
\noindent $N = 6 \then N^2 - 2 = 34 = 2 \times 17$, \\
\noindent $N = 7 \then N^2 - 2 = 47$; 素数,\\
\noindent $N = 8 \then N^2 - 2 = 62 = 2 \times 31$, \\
\noindent $N = 9 \then N^2 - 2 = 79$; 素数, \\
\noindent $N = 10 \then N^2 - 2 = 98 = 2 \times 7^2$, \\
\noindent $N = 11 \then N^2 - 2 = 119$, \\
\noindent $N = 12 \then N^2 - 2 = 142 = 2 \times 71$, \\
\noindent $N = 13 \then N^2 - 2 = 167$, \\
\noindent $N = 14 \then N^2 - 2 = 194 = 2 \times 97$, \\
\noindent $N = 15 \then N^2 - 2 = 253$. % }}}

(c) $N^2 - 3$の形の素数は多分無数に存在する. \\ % {{{
最初の方から正の数を列挙してみると

% TODO: 以下の表をtabularを使ってきれいに書く(noindentにすること)
$N = 2 \then 2^2 - 3 = 1$, \\
$N = 3 \then 3^2 - 3 = 6 = 2 \times 3$, \\
$N = 4 \then 4^2 - 3 = 13$; 素数, \\
$N = 5 \then 5^2 - 3 = 22 = 2 \times 11$, \\
$N = 6 \then 6^2 - 3 = 33 = 3 \times 11$, \\
$N = 7 \then 7^2 - 3 = 46 = 2 \times 23$, \\
$N = 8 \then 8^2 - 3 = 61$; prime, \\
$N = 9 \then 9^2 - 3 = 78 = 2 \times 39$, \\
$N = 10 \then 10^2 - 3 = 97$; prime

$N^2 - 4$の形の素数は5に限る.
なぜなら
\[
  N^2 - 4 = (N-2)(N+2)
\]
となるからである. %}}}

(d) 少なくとも平方数ではない.

% }}}

\cnt{1.5} % {{{

\noindent $n$が偶数のとき: \\
$n = 2m$ とおくと
\begin{align*}
1 + 2 + \cdots + 2m
  &= (2m + 1) + ((2m - 1) + 2) + ((2m - 2) + 3) + \cdots + ((m + 1 + m)) \\
  &= m \cdot (2m+1) \\
  &= \frac{n}{2}(n+1).
\end{align*}

\noindent $n$が奇数のとき: \\
$n = 2m - 1$とおくと
\begin{align*}
1 + 2 + \cdots + (2m - 1)
  &= (1 + (2m - 1)) + (2 + (2m - 2)) + \cdots + ((m - 1) + (m + 1)) + m \\
  &= \underbrace{2m + 2m + \cdots + 2m}_{m-1} + m \\
  &= 2m(m-1) + m \\
  &= (n+1)(\frac{n+1}{2} - 1) + \frac{n+1}{2} \\
  &= (n+1)(\frac{n+1}{2} - 1 + \frac{1}{2}) \\
  &= (n+1) \cdot \frac{n+1 - 2 + 1}{2} = \frac{n}{2}(n+1).
\end{align*}

% }}}

%}}}

\section{第2章の回答例} % {{{

\noindent \cnt{2.1} \indent % {{{
% TODO: 下の(a), (b)でenumerateするようにする
\noindent (a) 組み合わせとして考えるのは
\begin{align*}
(a, b) =& (3m, 3n), (3m+1, 3n), (3m+2, 3n), \\
        & (3m, 3n+1), (3m+1, 3n+1), (3m+2, 3n+1), \\
        & (3m, 3n+2), (3m+1, 3n+2), (3m+2, 3n+2)
\end{align*}
のように書かれる場合である.
必要なら$a$と$b$を入れ替えることによって, この表の右上半分だけを考える.
すなわち,
\begin{align*}
(a, b) = (3m+1, 3n+1), (3m+2, 3n+1), (3m+2, 3n+2)
\end{align*}
の場合 対応する既約ピタゴラス数$(a, b, c)$が存在しないことを示したい.

まず, $c = 3l$とかけているとき, $c^2$は3の倍数であり,
$c = 3l+1$とかけているとき, $c^2$は3で割ると1余り,
$c = 3l+2$とかけているとき, $c^2$は3で割ると1余る.

\noindent $(a, b) = (3m+1, 3n+1)$ のとき,
\begin{align*}
a^2 + b^2
  &= (3m+1)^2 + (3n+1)^2 \\
  &= 9(m^2 + n^2) + 6(m+n) + 2
\end{align*}
となるが, $c^2$は3で割ると2余る組がないから不適.

\noindent $(a, b) = (3m+2, 3n+1)$のとき,
\[
  a^2 + b^2 = 9(m^2 + n^2) + 6(2m + n) + 5.
\]
よって, この場合も$a^2 + b^2$を3で割ると2余るので不適.

\noindent $(a, b) = (3m+2, 3n+2)$のとき
\[
  (3m+2)^2 + (3n+2)^2 = 9(m^2 + n^2) + 12(m+n) + 8.
\]
よってこの場合も$a^2 + b^2$を3で割ると2余るので不適.

(b) 分からない.
% }}}

\cnt{2.2} % {{{
仮定より, ある整数$k_1, k_2$があって
\begin{align*}
  m &= dk_1, \\
  n &= dk_2
\end{align*}
が成り立つ.
このとき,
\begin{align*}
  m + n &= d(k_1 + k_2), \\
  m - n &= d(k_1 - k_2)
\end{align*}
となるから, 主張を得る.
% }}}

\cnt{2.3} \indent % {{{
% 以下の(a), (b)などはenumerateで書き換えること
(a) 任意の1より大きな奇数が現れる.
実際, 定理 2.1で$t=1$とおけば $a = s$.

(b) 少なくとも 4の倍数が現れる. 8で割って4余る数はかならず現れる.

$b$は4の倍数であることを示す.

$s, t$はともに奇数であるから, ある自然数$s_1, t_1$ があって, $s = 2s_1 - 1$, $t = 2t_1 - 1$
が成り立つ.
このとき
\begin{align*}
b
  &= \frac{1}{2}(s - t)(s + t) \\
  &= \frac{1}{2}(2s_1 - t_1)(2s_1 + 2t_1 - 2) \\
  &= 2(s_1 - t_1)(s_1 + t_1 - 1)
\end{align*}
となる.
ここで, $s_1 - t_1$ と $s_1 + t_1 - 1$は2を法として異なるので, $b$ は4の倍数である.

次に, 8で割って4余る数は必ず現れることを示す.

$b$を4で割って8余る自然数とする. ある互いに素な奇数$s, t (s > t)$ があって
\[
  b = \frac{s^2 - t^2}{2}
\]
を満たすことを示したい.
$b = 4k$($k$は奇数) と置くと
\[
  4k = \frac{s^2 - t^2}{2}.
\]
分母を払って
\[
  8k = s^2 - t^2.
\]
ここで
\begin{align*}
s^2
  &= t^2 + 8k \\
  &= t^2 + 8(k-2) + 16
\end{align*}
であるから, $t = (k-2)^2$と置くと
\begin{align*}
  s^2
    &= (k-2)^2 + 8(k-2) + 16 \\
    &= (k + 2)^2.
\end{align*}
よって, $s = k + 2$と置けばよい.
最後に, $k$は奇数であったから, $k-2$ と $k+2$ は互いに素である.

(c) 4で割ると1余る数.

$s = 2s_1 - 1$, $t = 2s_2 - 1$ と置くと
\begin{align*}
c
  &= \frac{s^2 + t^2}{2} \\
  &= \frac{1}{2} ((2s_1 - 1)^2 + (2s_2 - 1)^2) \\
  &= \frac{1}{2} (8(s_1^2 + s_2^2) - 8s_1s_2 + 2) \\
  &= 4(s_1^2 + s_2^2) - 4s_1s_2 + 1 \\
  &= 4(s_1^2 - s_1s_2 + s_2^2) + 1
\end{align*}
となるから, $c$は4で割ると1余る.
% }}}

\cnt{2.4} % {{{
見つけられない. % }}}

\cnt{2.5} % {{{
% 以下の a), b) をenumerate にすること.
\noindent a) % {{{
$n \le 4$における$a, b, c$の値を表にすると
% TODO:  以下の表をtabularを使うこと.
\begin{align*}
T_1 & = 1         & b_1 &= 4 \cdot 1  = 4 & 3^2 + 4^2  &= 5^2, \\
T_2 & = 1 + 2 = 3 & b_2 &= 4 \cdot 3 = 12 & 5^2 + 12^2 &=13^2, \\
T_3 & = 3 + 3 = 6 & b_3 &= 4 \cdot 6 = 24 & 7^2 + 24^2 &= 25^2, \\
T_4 & = 6 + 4 = 10 & b_4 &= 4 \cdot 10 = 40 & 9^2 + 40^2 &= 41^2.
\end{align*}
ここで, $b_n$を$T_n$に対応する既約ピタゴラス数の$b$とおいた.

\noindent $T_5)$ $T_5 = 10 + 5 = 15$. また, $b_5 = 4 \cdot 15 = 60$. % {{{

これより, $11^2 + 60^2 = 61^2$が予想できる.
実際計算してみると
\[
  11^2 + 60^2 = 121 + 3600 = 3721 = 61^2.
\]
% }}}

\noindent $T_6)$ % {{{
$T_6 = 15 + 6 = 21$, $b_6 = 4 \cdot 21 = 84$ より
\[
  13^2 + 84^2 = 85^2
\]
が予想できる.
計算してみると
\begin{align*}
13^2 + 84^2
  &= 169 + 7059 \\
  &= 7225, \\
85^2 &= 7225
\end{align*}
であるから, 確かに成り立っている.
% }}}

\noindent $T_7)$ % {{{
$T_7 = 21 + 7 = 28$, $b_7 = 4 \cdot 28 = 112$ であるから
\[
  15^2 + 112^2 = 113^2
\]
が予想される.
実際
\begin{align*}
15^2 + 112^2
  &= 225 + 12544 \\
  &= 12769, \\
113^2 &= 12769
\end{align*}
なので, 確かに成り立っている.
% }}}
% }}}

\noindent b) % {{{
存在する.
\[
  (a, b, c) = (2n+1, 4T_n, 4T_n + 1)
\]
が既約ピタゴラス数の組であることを示す.
実際, 定理 2.1において $t=1, s=2n+1$ とおけば $t$ と $s$ は互いに素で
\begin{align*}
a &= 2n + 1, \\
b
  &= \frac{(2n+1)^2 - 1}{2} \\
  &= \frac{4n^2 + 4n}{2} \\
  &= 2n(n+1) \\
  &= 4 \cdot \frac{n}{2} (n+1) = 4T_n, \\
c
  &= \frac{1^2 + (2n+1)^2}{2} \\
  &= \frac{4n^2 + 4n + 2}{2} \\
  &= 2n^2 + 2n + 1 \\
  &= 4 \cdot \frac{n}{2} (n+1) + 1 = 4T_n + 1.
\end{align*}
% }}}

% }}}

\cnt{2.6} % {{{

定理 2.1 において $c - a = 2$ とおけば
\[
  c - a = \frac{s^2 + t^2}{2} - st = 2
\]
より
\[
  s = t + 2
\]
を得る.

% TODO: 以下の (a), (b) をenumerateに

\noindent (a) % {{{

$t = 1, s = 3$ とおけば
\[
  a = 3, \quad
  b = \frac{9 - 1}{2} = 4, \quad
  c = \frac{3^2 + 1^2}{2} = 5.
\]
また, $t=3, s=5$ とおけば
\[
  a = 15,
  b = \frac{5^2 - 3^2}{2} = \frac{16}{2} = 8,
  c = \frac{3^2 + 5^2}{2} = 17.
\]
% }}}

\noindent (b) % {{{

差が2である自然数の組が共通因数を持つには, その2つの自然数が偶数であり,
共通因数は2である場合にかぎる.

例えば, $s = 999, t = 997$ とおくと
\[
  a = 999 \cdot 997,
  b = \frac{999^2 - 997^2}{2},
  c = \frac{999^2 * 997^2}{2}
\]
を得て, $c - a = 2$ である.
% }}}

\noindent (c) % {{{
$s = t + 2$ を定理 2.1に代入すればよい.
すなわち
\begin{align*}
  a &= t \cdot (t+2), \\
  b
    &= \frac{(t+2)^2 - t^2}{2} \\
    &= \frac{4t+4}{2} \\
    &= 2(t+1), \\
c
  &= \frac{t^2 + (t + 2)^2}{2} \\
  &= \frac{2t^2 + 4t + 4}{2} \\
  &= t^2 + 2t + 2.
\end{align*}
ここで, $t$は奇数.
% }}}

% }}}
% }}}

\end{document}
