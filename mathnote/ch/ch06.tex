\section{第6章の回答例} % {{{
\begin{enumerate}[6.1] % {{{
\item
  \begin{enumerate}[(a)]
    \item Step 1) $gcd(6, 15) = 3$ であるから, $6x + 15y$ は3の倍数全体を動く.
    よって, $6x + 15y = 3t$ とおき, まず
    \begin{align}
      3t + 20z = 1  \tag{$\ast$}
    \end{align}
    の解$(t, z)$を探す.
    ユークリッドの互除法より
    \begin{align}
      20 = 3 \times 6 + 2 \tag{1} \\
      3 = 2 \times 1 + 1  \tag{2} \\
      2 = 1 \times 2.  \tag{3}
    \end{align}
    $a = 3$, $b = 20$とおくと (1) より
    \begin{align*}
      b = 6a + 2. \\
      2 = b - 6a.
    \end{align*}
    (2)より
    \begin{align*}
      a = b - 6a + 1 \\
      6a - b = 1.
    \end{align*}
  よって, $(t, z) = (7, -1)$ が$(\ast)$の解の一つである.

  Step 2) 次に $6x + 15y = 3 \cdot 7 = 21$ すなわち
  \[
    2x + 5y = 7
  \]
  は$(x, y) = (1, 1)$ を解の一つとして持つ.

  以上によって, $(x, y, z) = (1, 1, -1)$ が解の一つである.

  \item ユークリッドの互除法より
  \begin{align}
    94321 &= 9876 \times 5   + 4941 \tag{1} \\
    9876  &= 4941 \times 1   + 4935 \tag{2} \\
    4941  &= 4935 \times 1   + 6    \tag{3} \\
    4935  &=    6 \times 822 + 3    \tag{4} \\
       6  &=    3 \times 2          \tag{5}
  \end{align}
  であるから, $\gcd(94321, 9876) = 3$ である.

  また, (1)より $a = 54321, b=9876$とおくと
  \begin{align*}
    4941 = a - 5b.
  \end{align*}
  (2)より
  \begin{align*}
    b &= a - 5b + 4935 \\
    4935 &= 6b - a.
  \end{align*}
  (3)より
  \begin{align*}
    a - 5b &= 6b - a + 6 \\
    2a - 11b &= 6.
  \end{align*}
  (4)より
  \begin{align*}
    6b - a &= (2a - 11b) \times 822 + 3 \\
    (6 + 11 \times 822)b - (1 + 2 \times 822)a &= 3 \\
    9048b - 1645a &= 3.
  \end{align*}
  以上によって $(a, b) = (-1645, 9048)$は1つの解である.

  (5)より
  \begin{align*}
    -8a + 7b &= (15a - 13b) \times 3 + 1 \\
    -53a + 46b &= 1.
  \end{align*}
  よって, $(x, y) = (-53, 46)$は一つの方程式の解である.

  また, $g = \gcd(105, 121) = 1$ であるから, 一般会は
  \[
    x = -53 + 121k, \quad
    y =  46 - 105k \quad (k \in \ZZ)
  \]
  である.
  \end{enumerate}
  \item
  \begin{enumerate}[(a)]
    \item
    ユークリッドの互除法より
    \begin{align}
      121 &= 105 \times 1 + 16 \tag{1} \\
      105 &=  16 \times 6 + 9  \tag{2} \\
      16  &=   9 \times 1 + 7  \tag{3} \\
      9   &=   7 \times 1 + 2  \tag{4} \\
      7   &=   2 \times 3 + 1  \tag{5} \\
      2   &=   1 \times 2      \tag{6}
    \end{align}
    であるから, $\gcd(121, 105) = 1$ である.

    $b = 121, a = 105$ とおくと (1)より
    \begin{align*}
      b  & = a + 16 \\
      16 & = b - a.
    \end{align*}
    (2)より
    \begin{align*}
      a &= (b - a) \times 6 + 9 \\
      9 &= 7a - 6b.
    \end{align*}
    (3)より
    \begin{align*}
      b - a &= 7a - 6b + 7 \\
      7 &= -8a + 7b.
    \end{align*}
    (4)より
    \begin{align*}
      7a - 6b &= -8a + 7b + 2 \\
      2 &= 15a - 13b.
    \end{align*}
    (5)より
    \begin{align*}
      -8a + 7b &= (15a - 13b) \times 3 + 1 \\
      -53a + 46b &= 1.
    \end{align*}
    よって, $(x, y) = (-53, 46)$は一つの方程式の解である.

    また,$g = \gcd(105, 121) = 1$であるから, 一般解は
    \[
      x = -53 + 121k, \quad
      y = 46 - 105 \quad
      (k \in \ZZ)
    \]
    である.
  \item
    Ex 6.1.aと同じ方程式である.
    一般解は
    \begin{align*}
      (x, y)
        &= (11, -2) + \frac{k}{15} (67890, -12345) \\
        &= (11, -2) + k(4526, -823)
    \end{align*}
    である.
  \item
    Ex6.1.bと同じ方程式である.
    一般解は
    \begin{align*}
      (x, y)
        &= (-1645, 9048) + \frac{k}{3}(9876, -54321) \\
        &= (-1645, 9048) + k(3292, -18107)
    \end{align*}
    である.
  \end{enumerate}
\item
  \begin{enumerate}[(a)] \item
    Step 1) $\gcd(6, 15) = 3$ であるから, $6x + 15y$ は$3$の倍数全体を動く.
    よって, $6x + 15y = 3t$ とおき, まず
    \begin{align}
      3t + 20z = 1 \tag{$\ast$}
    \end{align}
    の解 $(t, z)$ を探す.
    ユークリッドの互除法より
    \begin{align}
      20 &= 3 \times 6 + 2 \tag{1} \\
      3 &= 2 \times 1 + 1 \tag{2} \\
      2 &= 1 \times 2. \tag{3}
    \end{align}
    $a = 3, b = 20$ とおくと (1)より
    \begin{align*}
      b &= 6a + 2 \\
      2 &= b - 6a.
    \end{align*}
    (2)より
    \begin{align*}
    a &= b - 6a + 1 \\
    7a - b &= 1.
    \end{align*}
    よって, $(t, z) = (7, -1)$ は $(\ast)$の解の一つである.

  Step 2)
    次に $6x + 15y = 3 \cdot 7 = 21$ すなわち
    \[
      2x + 5y = 7
    \]
    は $(x, y) = (1, 1)$ を解の一つとして持つ.

    以上によって, $(x, y, z) = (1, 1, -1)$ が解の一つである.
  \end{enumerate}
\end{enumerate} % }}}
% }}}

