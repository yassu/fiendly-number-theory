\section{第4章の回答例} % {{{

\cnt{4.1} % {{{
略
% }}}

\cnt{4.2} \indent % {{{
\noindent (a) $a = xz, b = yz, c = z^2$ とおくと % {{{
\begin{align*}
  x^3 z^3 + y^3 z^3 = z^4 \\
  x^3 + y^3 = z.
\end{align*}
したがって, 例えば
\begin{itemize}
\item $x = 1, y = 2$ のとき $z = 1^3 + 2^3 = 9$ で
\[
  a = 1 \cdot 9 = 9, \quad
  b = 2 \cdot 9 = 18, \quad
  c = 9^2 = 81.
\]
\item $x = 2, y = 3$ のとき $z = 2^3 + 3^3 = 35$ で
\[
  a = 2 \cdot 35 = 70, \quad
  b = 3 \cdot 35 = 105, \quad
  c = 35^2.
\]
\end{itemize}
% }}}

\noindent (b) % {{{
仮定より
\[
  A^3 + B^3 = C^2.
\]
このとき
\begin{align*}
(n^2A)^3 + (n^2B)^3
  &= n^6(A^3 + B^3) \\
  &= n^6 C^2 \\
  &= (n^3C)^2.
\end{align*}
% }}}

\noindent (c) % {{{

(a)の$x, z$が平方因子を持たず かつ互いに素であれば $a = xz$ に平方因子は現れないから
$(a, b, c)$ は $(b)$の意味の既約である.

以下のステップで解を見つける:

\begin{enumerate}[(i)]
\item $x = 1, y$を適当な平方因子を持たない自然数として $z = x^3 + y^3$ を計算する.
\item $z$ が平方因子を持たなければ $(a, b, c) = (xz, yz, z^2)$ は 既約である
  \footnote{xzに平方因子が現れないため}.
\end{enumerate}

$x=1, y=2$ のとき) $z = 1^3 = 9^3 = 9 = 3^2$. これは平方因子を持つ.

$x=1, y=3$ のとき) $z = 1^3 + 3^3 = 28 = 2^2 \cdot 7$. これは平方因子を持つ.

$x=1, y=4$ のとき) $z = 1^3 + 4^3 = 65 = 5 \times 13$. これは平方因子を持たない.

$x=1, y=5$ のとき) $z = 1^3 + 5^3 = 126 = 2 \times 63 = 2 \times 7 \times 3^2$.
  これは平方因子を持つ.

$x=1, y=6$のとき) $z = 1^3 + 6^3 = 217 = 7 \times 31$. これは平方因子を持たない.

$x=1, y=7$のとき)
\begin{align*}
z
  &= 1^3 + 7^3 \\
  &= 344 \\
  &= 2^2 \times 86.
\end{align*}
これは平方因子を持つ.

$x = 1, y=8$のとき)
\begin{align*}
z
  &= 1^3 + 8^3 \\
  &= 513 \\
  &= 3^2 \times 57.
\end{align*}
これは平方因子を持つ.

$x = 1, y = 9$ のとき $z = 1^3 + 9^3 = 730 = 2 \times 3 \times 73$. これは平方因子を持たない.

以上によって,
\begin{align*}
(a, b, c)
  &= (2, 2, 4), (65, 4 \times 65, 65^2), \\
  & \quad (7 \times 31, 6 \times 7 \times 31, 7^2 \times 31^2),
      (2 \times 3 \times 73, 2 \times 3^3 \times 73, 2^2 \times 3^2 \times 73^2)
\end{align*}
は既約な $(\ast)$ の解である.
% }}}

\noindent (d) % {{{
式 $(\ast)$ で$a = b$とした方程式
\[
  2a^3 = c^2
\]
を考える.
左辺は偶数だから, $c$ も偶数. よって 自然数 $c_1$ があって $c = 2c_1$ と書ける.
このとき
\begin{align*}
2a^3 &= (2c_1)^2 \\
2a^3 &= 4c_1^2 \\
a^3 &= 2c_1^2.
\end{align*}
同様に $a$は偶数なので, 自然数 $a_1$ があって $a = 2a_1$.
このとき
\begin{align*}
  8a_1^3 &= 2c_1^2 \\
  4a_1^3 &= c_1^2.
\end{align*}

同様に$c_1$は偶数なので, 自然数$c_2$があって $c_1 = 2c_2$.
このとき
\begin{align*}
4a_1^3 &= 4c_2^2. \\
a_1^3 &= c_2^2.
\end{align*}
よって ある自然数$n$があって $a_1 = n^2, c_2 = n^3$.
したがって
\begin{align*}
a &= 2a_1 = 2n^2, \\
c &= 2c_1 = 4c_2 = 4n^3.
\end{align*}
以上によって, $(a, b, c) = (2n^2, 2n^2, 4n^3)$($n$は自然数).
% }}}

\noindent (e)
$x = 1$ とし
\[
  1 + y^3 = 1^3 + y^3 > 10^4
\]
とすれば $y^3 > 10^4$ であるが, これをみたす自然数$y$は$y \ge 22$を満たす \footnote{
  $10^{4/3} = 21.544\ldots$のため}

$y=22$ のとき: $z = 1^3 = 22^3 = 10649 = 23 \times 463$ であるから (c)の条件を満たす.
  以上によって
  \[
  (a, b, c) = (23 \times 463, 22 \times 23 \times 463, 23^2 \times 463^2)
  \]
  は$a > 10^4$を満たす$(\ast)$の既約な解である.
% }}}
% }}}

