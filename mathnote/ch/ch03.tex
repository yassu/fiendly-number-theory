\section{第3章の回答例} % {{{

\cnt{3.1} % {{{

\noindent (a) % {{{
$u$と$v$が共通因数$k > 1$をもつとする.
このとき, $u = ku^\prime, v = kv^\prime$とおくと
\[
  a = k^2(u^{\prime 2} - v^{\prime 2}), \quad
  b = 2k^2 u^{\prime 2} v^{\prime 2}, \quad
  c = k^2 (u^{\prime 2} - v^{\prime 2})
\]
となるから, $a, b, c$は共に$k^2$で割り切れる.
% }}}

\noindent (b) % {{{
$u$, $v$が共に奇数であれば, $u^2 - v^2$, $u^2 + v^2$は偶数であるから, $(a, b, c)$は
  既約ピタゴラス数の組でない.

$u=3, v=1$とおくと
\[
  (a, b, c) = (8, 6, 10)
\]
となるが, これは既約ピタゴラス数の組ではない.
% }}}

\noindent (c) % {{{
(d)の問題のために, $s$と$t$が共通因数を持つ場合は右に$\checkmark$を,
$s$と$t$が共通因数を持たないが$(a, b, c)$が既約ピタゴラス数の組ではない場合は$\ccheckmark$をつける.

\begin{longtable}{llllll}
u & v & a & b & c & \\
2 & 1 & 3 & 4 & 5 & \\
3 & 1 & 8 & 6 & 10 & \ccheckmark \\
3 & 2 & 5 & 12 & 13 & \\
4 & 1 & 15 & 8 & 17 & \\
4 & 2 & 12 & 16 & 20 & \checkmark \\
4 & 3 & 7 & 24 & 25 & \\
5 & 1 & 24 & 10 & 26 & \ccheckmark \\
5 & 2 & 21 & 20 & 29 & \\
5 & 3 & 16 & 30 & 34 & \ccheckmark \\
5 & 4 & 9 & 40 & 41 & \\
6 & 1 & 35 & 12 & 37 & \\
6 & 2 & 32 & 24 & 40 & \checkmark \\
6 & 3 & 27 & 36 & 45 & \checkmark \\
6 & 4 & 20 & 48 & 50 & \checkmark \\
6 & 5 & 11 & 60 & 61 & \\
7 & 1 & 48 & 14 & 50 & \ccheckmark \\
7 & 2 & 45 & 28 & 53 & \\
7 & 3 & 40 & 42 & 58 & \ccheckmark \\
7 & 4 & 33 & 56 & 65 & \\
7 & 5 & 24 & 70 & 74 & \ccheckmark \\
7 & 6 & 13 & 84 & 85 & \\
8 & 1 & 63 & 16 & 65 & \\
8 & 2 & 60 & 32 & 68 & \checkmark \\
8 & 3 & 55 & 48 & 73 & \\
8 & 4 & 48 & 64 & 80 & \checkmark \\
8 & 5 & 39 & 80 & 89 & \\
8 & 6 & 28 & 96 & 90 & \checkmark \\
8 & 7 & 15 & 112 & 113 & \\
9 & 1 & 80 & 18 & 82 & \ccheckmark \\
9 & 2 & 77 & 36 & 85 & \\
9 & 3 & 72 & 54 & 90 & \checkmark \\
9 & 4 & 65 & 72 & 97 & \\
9 & 5 & 56 & 90 & 106 & \ccheckmark \\
9 & 6 & 45 & 108 & 117 & \checkmark \\
9 & 7 & 32 & 126 & 130 & \ccheckmark \\
9 & 8 & 17 & 144 & 145 & \\
10 & 1 & 99 & 20 & 101 & \\
10 & 2 & 96 & 40 & 104 & \checkmark \\
10 & 3 & 91 & 60 & 109 & \\
10 & 4 & 84 & 80 & 116 & \checkmark \\
10 & 5 & 75 & 100 & 125 & \checkmark \\
10 & 6 & 64 & 120 & 136 & \checkmark \\
10 & 7 & 51 & 140 & 149 & \\
10 & 8 & 36 & 160 & 164 & \checkmark \\
10 & 9 & 19 & 180 & 181 &
\end{longtable}
% }}}

\noindent (d) % {{{
「$u$と$v$が共通因数を持たない」かつ「$u$と$v$の偶奇が異なる」.
% }}}

\noindent (e) % {{{
$u$と$v$が共通因数を持つ場合, $(a, b, c)$は既約ではないことは 3.1で示した.
$u$と$v$が奇数であるとき, $u^2 - v^2$は偶数であり, $2uv$も偶数であるから, $(a, b, c)$
  は既約でない.

$u$と$v$が共通因数を持たず, $u$と$v$の偶奇が異なるとき $u^2 -
v^2$と$2uv$が互いに素であることを示そう.

$u^2 - v^2$と$2uv$が十数$p$を共通因数として持ったとして, $2uv = pu^\prime, u^2-v^2 =
pv^\prime$とおく.

$u^2 - v^2$は奇数であるから, $p$は奇素数である.
また, $2uv=pu^\prime$より, $u$か$v$は$p$の倍数である.

よって, $u$も$v$も$p$の倍数であり, これは$u$と$v$が共通因数を持たないことに反する.
% }}}

% }}}

\cnt{3.2} % {{{

\noindent (a) % {{{
円の方程式$x^2 + y^2 = 2$に傾き$m \in \QQ$で$(1, 1)$を通る直線の方程式
$y = m(x-1) + 1 = mx - m + 1$を代入すると
\begin{align*}
x^2 + (m(x-1) + 1)^2 = 2 \\
x^2 + (mx - m + 1)^2 = 2 \\
(1 + m^2)x^2 + 2mx(1 - m) + m^2 - 2m - 1 = 0 \\
(x - 1)((1 + m^2)x - (m^2 - 2m - 1)) = 0.
\end{align*}
よって, $x \neq 1$とすると
\[
  x = \frac{m^2 - 2m - 1}{1 + m^2}.
\]
これを $y = mx - m + 1$に代入して
\begin{align*}
y
  &= m \cdot \frac{m^2 - 2m - 1}{1 + m^2} - m + 1 \\
  &= \frac{m(m^2 - 2m - 1) - m (1 + m^2) + 1 + m^2}{1 + m^2} \\
  &= \frac{-m^2 - 2m + 1}{1 + m^2} \\
  &= - \frac{m^2 + 2m - 1}{1 + m^2}.
\end{align*}

最後に $m = \frac{v}{u}$ とおくと
\begin{align*}
  x
    &= \frac{v^2 - 2uv - u^2}{u^2 + v^2}, \\
  y
    &= - \frac{v^2 + 2uv - u^2}{u^2 + v^2} \\
    &= \frac{u^2 - 2uv - v^2}{u^2 + v^2}.
\end{align*}

% }}}

\noindent (b) % {{{
$x^2 + y^2 = 3$上の点$(x_0, y_0) \in \QQ^2$を見つけられない.
% }}}
% }}}

\cnt{3.3} % {{{

$x^2 + y^2 = 1$に$(-1, 0)$を通り $m \in \QQ$を傾きとする直線の方程式$y = m(x+1)$を連立させると
\begin{align*}
x^2 - m^2(x + 1)^2 = 1. \\
(1 - m^2)x^2 - 2m^2x - 1 - m^2 = 0. \\
(x + 1)((1-m^2)x - (1+m^2)) = 0.
\end{align*}

$x \neq -1$ とすると
\[
  x = \frac{1 + m^2}{1 - m^2}.
\]
これを $y = m(x+1)$に代入すると
\begin{align*}
y
  &= m(x+1) \\
  &= m \cdot (\frac{1+m^2}{1-m^2} + 1) \\
  &= m \cdot \frac{1 + m^2 + 1 - m^2}{1 - m^2} \\
  &= \frac{2m}{1 - m^2}.
\end{align*}

また, $m = \frac{v}{u}$とおくと
\[
  x = \frac{u^2 + v^2}{u^2 - v^2}, \quad
  y = \frac{2uv}{u^2 - v^2}.
\]
% }}}

\cnt{3.4} % {{{
TODO
% }}}

% }}}

